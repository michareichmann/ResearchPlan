\head{Research Topics}

Within the RD42 Collaboration \me is investigating the behaviour the signal response of planar and 3D \ac{pCVD} diamond detectors in pad or pixel geometries depending on incident particle flux. He is also characterising the composition of 3D detectors and the internal structure of the \ac{pCVD} diamond material with a high resolution beam telescope.\par
The high rate tests are performed at \ac{PSI} using the beam line piM1 with a positive \SI{260}{\mega\electronvolt\per c} pion beam and tunable particle fluxes from \SI{1}{\kilo\hertz\per\centi\meter^2} up to \SI{20}{\mega\hertz\per\centi\meter^2} whereas the high resolution test are performed at CERN using the SPS beam line H6 with pions or protons up to momenta of \SI{200}{\giga\electronvolt\per c}.\parsmall

% HIGH RATE BEAM TEST
\me is in charge of organising and conducting the RD42 high rate beam tests at \ac{PSI}. This includes the maintenance and further development of the ETH beam telescope and the overall set-up at \ac{PSI}, qualifying the data-taking and improving as well as adapting the \ac{DAQ} framework EUDAQ.\par

% PAD
In order to proof that \ac{pCVD} diamond material is suited as a particle detector at high rates, the most simple detector geometry - a pad detector - is investigated. The detectors which are built at \ac{OSU} by metallising them with a thin layers on both sides, are measured at various fixed rates recording their signals as digital information. These waveforms are then analysed to get conclusions about their behaviour at the tested rates.\par

% PIXEL
Since the current technology at the innermost tracking layers of the LHC is based on planer pixel detectors also \ac{pCVD} diamond sensors read out with state of the art pixel detectors are investigated. The \acp{ROC} digitise the signals above a tunable threshold already on chip using an \ac{ADC}. Therefore also the effects of threshold as well as the more complex electric field in the pixel geometry are analysed.\par

% 3D
3D detectors are a very promising radiation tolerant detector concept. Within RD42 these detectors are built in collaboration with the Universities of Manchester, Oxford and Ohio utilising a femtosecond laser. \me is testing the general working principal as well as the rate behaviour of these detectors in both pad-like and pixel geometry.\parsmall

% HIGH RESOLUTION BEAM TEST
The \ac{pCVD} diamond is a material consisting of many single crystalline cells along the direction of growth which vary in size and shape. This fact also plays an important role for the detectors, especially 3D. Therefore the current ETH telescope is upgraded in order to measure both the inner structure of the material as well as the structure of the various detector geometries and designs with a high spatial resolution.\parend