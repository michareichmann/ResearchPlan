\head{Research Topics}

Within the RD42 Collaboration Mr. Reichmann is investigating the behaviour the signal response of planar and 3D \ac{pCVD} diamond detectors in pad or pixel geometries depending on incident particle flux and characterising their internal structure with a high resolution beam telescope. The rate tests are solely performed at \ac{PSI} using the beam line piM1 with a positive \SI{260}{\mega\electronvolt\per c} pion beam and tunable particle fluxes from \SI{1}{\kilo\hertz\per\centi\meter^2} up to \SI{10}{\mega\hertz\per\centi\meter^2} whereas the high resolution test are performed at CERN using the SPS beam H6 with pions or protons up to momenta of \SI{200}{\giga\electronvolt\per c}.\parsmall

\headsmall{Rate Beam Tests}
Mr. Reichmann is in charge of organising and conducting the RD42 high rate beam tests at \ac{PSI} as well as assisting other members of the collaboration performing experiments at this facility. This includes maintaining and operating the ETH beam telescope and the overall set-up at \ac{PSI}, supervising the data-taking and running and improving the \ac{DAQ} framework EUDAQ.\parsmall

\headsmall{Pad Detectors}
As a proof of principle that \ac{pCVD} diamond material is suited as particle detectors at high particle rates, the most simple detector geometry - pad detectors - are investigated. The investigated detectors are built at \ac{OSU} by cleaning diamonds from the company II-IV and metallising them with a thin Cr-Au layer on both sides. They will then be measured at \ac{PSI} at various fixed rates recording their signals as digital waveforms which are then to be analysed to get conclusions about their behaviour at different rates.\parsmall

\headsmall{Pixel Detectors}
Since the current technology at the innermost tracking layers of the LHC is based on planer pixel detectors it also very important that Mr. Reichmann will investigate \ac{pCVD} diamond sensors read out with state of the art pixel detectors which digitise the signals already on chip using an \ac{ADC}. \parsmall

\headsmall{3D Detectors}
3D detectors are a very promising detector concept. Within RD42 these detectors are built in collaboration with the Universities of Manchester, Oxford and Ohio utilising a femtosecond laser. Mr. Reichmann will test the general working principal as well as the rate behaviour of these detectors both pad-like and pixel geometry.\parsmall

\headsmall{High Resolution Beam Tests}
As the name already suggests \ac{pCVD} material is not uniform and therefore Mr. Reichmann shall also upgrade the current ETH telescope in order to measure the various detector geometries and designs with high resolution and resolve their inner structure.\parend