\head{The Collaboration}

The RD42 Collaboration at CERN investigates \ac{CVD} diamond as a future material for high energy particle detectors and is leading an effort to develop a tracking detector that can be operated in the extremely high radiation environment of the LHC's (and the HL-LHC's) innermost layers close to the beam pipe. Its advantageous properties make diamond a suitable material for such detector applications.\par
%Starting from the close relation and constant exchange with the manufactures of artificial diamonds which grow diamonds using a \ac{CVD} process 
The collaboration is investigating the bulk and surface properties of this interesting material and develops own procedures on preparing the diamonds for different detectors geometries and designs which are namely pad, pixel and 3D detectors to address specific issues related to their use at the LHC. Theses prototypes are then qualified using various different testing methods including long time current and irradiation studies as well as the investigation of their signal behaviour depending on incident particle flux and high resolution studies of the detector structure.\par
During the last few years the RD42 group has already shown that diamonds are radiation tolerant up to a fluence of \SI{2e16}{hadrons\per\centi\meter^2} and thus can operate for several years in the environment of the HL-LHC. They also do not show evidence of any damage due to electrons and photons up to \SI{100}{\mega\rad}.\parend