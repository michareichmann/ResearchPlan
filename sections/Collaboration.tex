\head{The Collaboration}

This dissertation is done within the CERN's RD42 Collaboration which investigates diamond as a future material for high energy particle detectors. Aiming to build a fully functional tracking detector that can be operated in the extremely high radiation environment of the LHC's (and the HL-LHC's) innermost layers close to it's beam pipe the RD42 Collaboration is involved in every process leading to that purpose.\par
Starting from the close relation and constant exchange with the manufactures of artificial diamonds which grow diamonds using a \ac{CVD} process the collaboration is investigating the bulk and surface properties of this interesting material and develops own procedures on preparing the diamonds for different detectors designs which are namely are pad, pixel and 3D detectors. Theses prototypes are then qualified using various different testing methods including long time current and irradiation studies as well as the investigation of their signal behaviour depending on incident particle flux and high resolution studies of the detector structure.\par
The collaboration has already shown that diamonds are radiation tolerant up to a fluence of \SI{1e16}{hadrons\per\centi\meter^2} and can operate for several years in the environment of the HL-LHC and that they do not show evidence of any damage due to electrons and photons up to \SI{100}{\mega\rad}.\parend